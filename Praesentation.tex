\documentclass[german,10pt,xcolor=colortbl,compress]{beamer}%,draft]{beamer}
\usepackage[utf8]{inputenc}
\usepackage[OT1]{fontenc}
\usepackage{calc}
\usepackage[ngerman]{babel} % Neue Rechtschreibung
\usepackage{amsmath,amsthm,amssymb,euscript} % AMS-LaTeX  
\usepackage{enumerate,graphicx}

\usepackage{sansmathaccent}
%\pdfmapfile{+sansmathaccent.map}
\usepackage{calc}
\usepackage{movie15}
\usepackage{subfigure}
\usepackage{wasysym}
\usepackage{moresize}
\usepackage{dsfont}
\usepackage[misc]{ifsym} % for the letter picture before email on last slide

\newcommand{\fT}{\mathcal{T}}
\newcommand{\ZZ}{\mathbb{Z}}
\newcommand{\NN}{\mathbb{N}}
\newcommand{\PP}{\mathbb{P}}
\DeclareMathOperator{\argmin}{argmin}


% Load Theme
%%%%%\usetheme[noptsans,navigation=true, FB=Mathematik, frametotal=true]{TUKL}
%

\beamerdefaultoverlayspecification{<+->}

\title{Fachpraktikum}
\subtitle{Primzahltests modifiziert zum Testen von Polynomen auf Irreduzibilität}
\date[]{14.08.2019} %\\[1ex] OCIP 2016
%\author[Claudia Totzeck]{Claudia Totzeck}
%\institute[]{AG Technomathematik\\FB Mathematik\\TU Kaiserslautern}
%Setze ein Logo auf der Titelseite unten rechts
%%%%%\renewcommand{\theSecondLogo}{}
\providecommand{\norm}[1]{\lVert#1\rVert}
\newcommand{\dd}{\,\mathrm{d}}
\newtheorem{proposition}{Proposition}

\setbeamercolor*{block title example}{bg=lightgray}
\setbeamercolor*{block body example}{fg= black, bg= white}

%%%%%\setbeamercolor{block title}{bg=tuklblue,fg=white}

\begin{document}
	\maketitle
		
	\begin{frame}{Einleitung}	
		\begin{itemize}
			\item Primzahltests untersuchen: Welche Eigenschaften werden genutzt?
			\item Übertragbarkeit auf Polynome über $\ZZ_q$ bei festem $q \in \PP$?	
		\end{itemize}
	\end{frame}
	
	
	\begin{frame}[<*>]{Fermat}
		\begin{block}{Satz von Fermat}
			Ist $p$ eine Primzahl, so gilt für alle $ a \in \NN \text{ mit } p \nmid a:  
			\newline	
			a^{p-1} \equiv 1 \text{ mod } p $	
		\end{block}
	
		\medskip
		\uncover<2->{
		Algebra: $|(\ZZ_p)^*| = p-1$} \\
		\medskip
		\uncover<3->{
		Polynome: $ |(\ZZ_q[x]/f)^*| = q^{deg(f)}-1 $ für irreduzible Polynome $f$}	
	\end{frame}
	
	
	\begin{frame}[<*>]{Fermat}
		\begin{block}{Fermat für Polynome}
			Ist $f$ irreduzibel über $\ZZ_q$, so gilt für alle $ a \in \ZZ_q[x] \text{ mit } f \nmid a:
			 \newline a^{q^{deg(f)}-1}  \equiv 1 \text{ mod } f$
		\end{block}
		\medskip
		\uncover<2->{
		Als Test auf Irreduzibilität: Gilt $a^{q^{deg(f)}-1}  \not\equiv 1 \text{ mod } f$, dann ist $f$ nicht irreduzibel.}
	\end{frame}


	\begin{frame}[<*>]{Carmichael-Polynome}
		\begin{block}{Definition}
			Ein \emph{Carmichael-Polynom} ist ein zusammengesetztes Polynom $f$, sodass $a^{q^{deg(f)}-1}  \equiv 1 \text{ mod } f \text{ für alle } a \in \ZZ_q[x] \text{ mit } deg(ggT(a,f))=0$
		\end{block}
		\uncover<2->{
		\begin{block}{Satz}
			Sei $f\in \ZZ_q[x]$. Wenn für alle $f_i$ irreduzibel mit $f_i|f$ gilt, dass $f_i^2 \nmid f$ und $deg(f_i)|deg(f)$, dann ist $f$ ein Carmichael-Polynom.
		\end{block}}
		\vspace{0.5cm}
		\uncover<3->{
		$\Rightarrow$ false-positives einfach zu finden}
	
	%haben versucht, den Anteil Carmichael-Polynome zu bestimmen/abzuschätzen; hat nicht so geklappt.
	\end{frame}


	\begin{frame}{Miller-Rabin}
		\begin{itemize}
			\item Finde $s,u \in \NN,u $ ungerade mit $p-1=2^su $
			\item Wähle $a \in \NN$ 
			\item Teste, ob $a^u \equiv 1 \text{ mod } p$
			\item Für $1\leq t < s \text{ teste, ob } a^{2^t u} \equiv -1 \text{ mod } p$
		\end{itemize}
	\end{frame}


	\begin{frame}{Miller-Rabin für Polynome}
		\begin{itemize}
			\item Finde $s,u \in \NN,u $ ungerade mit $q^{deg(f)}-1=2^su $
			\item Wähle $a \in \ZZ_q[x]$ 
			\item Teste, ob $a^u \equiv 1 \text{ mod } f$
			\item Für $1\leq t < s \text{ teste, ob } a^{2^t u} \equiv -1 \text{ mod } f$	
		\end{itemize}		
	\end{frame}


	\begin{frame}{Schwierigkeiten}
		Laufzeit: 
		
		\begin{itemize}
			\item Sehr viele Allokationen; gelöst durch In-place-rechnen
			\item Potenzierung langsam, da $u$ oft groß
		\end{itemize}
	\end{frame}


	\begin{frame}{Power-Residue Symbol}
		Legendre Symbol für Polynome 
		\begin{block}{Definition}
			Für $d|q-1 \text{ fest},  a,f \in \ZZ_q[x], f \text{ irreduzibel}, f \nmid a: 
			\newline 	
			(\frac{a}{f})_d  \equiv a^{\frac{|f|-1}{d}} \text{ mod } f$
		\end{block}
	
		\begin{block}{Reziprozitätsgesetz}
			Seien $f,g$ irreduzible Polynome. Dann gilt:
			$(\frac{g}{f})_d=(-1)^{deg(f)deg(g)\frac{q-1}{d}} \cdot (\frac{f}{g})_d $ 				
		\end{block}
	\end{frame}

	
	\begin{frame}{Jacobi Symbol}
		Verallgemeinerung des Power-Residue Symbols: $f$ muss nicht irreduzibel sein.
		
		\begin{block}{Reziprozitätsgesetz}
			Seien $f,g$ teilerfremde Polynome, $q$ die Charakteristik von $\ZZ_q[x]$ und $d$ ein Teiler von $q-1$.
			$sgn(f):=lc(f)^{\frac{q-1}{d}}$ 
			Dann gilt:
			$\left(\frac{f}{g}\right) \cdot \left(\frac{g}{f}\right)^{-1} = (-1)^{\frac{q-1}{d} \cdot deg(f)\cdot deg(g)}\cdot sgn(f)^{deg(g)} \cdot sgn(g)^{-deg(f)} $
							
		\end{block}
	\end{frame}

	
	\begin{frame}{Power-Residue Test}
		\begin{itemize}
		\item Nutze Reziprozitätsgesetz, um $(\frac{a}{f})_d $ zu berechnen
		\item Vergleiche Ergebnis mit der Definition
		\end{itemize}
	\end{frame}

	
	\begin{frame}{Laufzeit}
		\begin{itemize}
			\item Ein Durchlauf sehr schnell; vergleichbar mit \textit{isirreducible}
			\item Problem: gibt oft fälschlicherweise \texttt{true} aus
			\item Abhängig von a
		\end{itemize}
	\end{frame}

	
	\begin{frame}{Pocklington}
		\begin{block}{Pocklington Kriterium}
			Sei $N\in \NN_{>1}$. $%\exists a, p \in \NN
			\text{Sei } a \in \NN \text{, s.d. } a^{N-1} \equiv 1 \text{ mod } N$.\\
			Sei $p$ prim, $p | N-1$ und $p> \sqrt{N}-1$.\\
			Wenn $ggT(a^{\frac{N-1}{p}}-1,N) = 1$, dann ist $N$ eine Primzahl.
		\end{block}
	\end{frame}

		
	\begin{frame}{Pocklington}
		\begin{block}{Pocklington für Polynome}
			Sei $f$ das zu testende Polynom und $a$ ein Polynom, s.d. 
			$q$ Charakteristik des Rings, $d$ der Grad von $f$.
			Falls $a^{q^{d}-1}\equiv 1 \text{ mod }f$ und 
			$\exists p \in [q^{\frac{d}{2}}, \frac{q^{d-1}}{2}]\text{, } p \text{ prim, } p|q^{d-1} $ : $ggT(a^{\frac{q^{d-1}}{p}}-1, f)=1$, dann ist $f$ irreduzibel. 
		\end{block}
	
	%Das Blöde ist, ein solches $p$ existiert nicht immer; dafür ist gut, dass wenn es mal funktioniert, hat man ein Zertifikat, mit dem man leicht nachweisen kann, dass es prim/irreduzibel ist.
	%$\Rightarrow$ tatsächlich ein Primzahltest/Irreduzibilitätstest.
	\end{frame}


	\begin{frame}{Laufzeit}
		\item Ein Durchlauf sehr schnell; vergleichbar mit \textit{isirreducible} 
			
		\item Schnell durch In-Place-Rechnung
			
		
	\end{frame}


	\begin{frame}[<*>]{Lucas-Folgen}
		Lineare Rekursionsgleichung $(a_n)_n$ von Grad 2
		\begin{block}{Satz}
			$\chi_c$ irreduzibel $\Rightarrow per(c) \bigl| |K|^2-1 = (q^{deg(f)})^2-1$
		\end{block}
		\medskip
		\uncover<2->{
		Als Test auf Irreduzibilität: $a_{per} \neq a_0 \text{ oder } a_{per+1} \neq a_1 \Rightarrow f$ nicht irreduzibel.}\\
		\medskip
		\uncover<3->{
		Verschiedene Möglichkeiten $a_{per}$ zu berechnen:}
		\begin{itemize}
			\uncover<4->{
			\item rekursiv $\Rightarrow$ Laufzeit!}
			\uncover<5->{
			\item explizit mit Matrix}
			\uncover<6->{
			\item mit Lucas-Kette: bestimmte Form der Rekursionsgleichung gegeben, dafür einfache Formel, die Glieder explizit auszurechnen; rechnen im Ring} \uncover<7->{$((\ZZ_q[t]/f)[s])/(s^2-a \cdot s+b)$}
			%$a,b \in \ZZ_q[t]$
		\end{itemize}
	\end{frame}
	
\end{document}
