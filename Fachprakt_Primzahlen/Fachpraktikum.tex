\documentclass[german,10pt,xcolor=colortbl,compress]{beamer}%,draft]{beamer}
\usepackage[utf8]{inputenc}
\usepackage[OT1]{fontenc}
\usepackage{calc}
\usepackage[ngerman]{babel} % Neue Rechtschreibung
\usepackage{amsmath,amsthm,amssymb,euscript} % AMS-LaTeX  
\usepackage{enumerate,graphicx}

\usepackage{sansmathaccent}
\pdfmapfile{+sansmathaccent.map}
\usepackage{calc}
\usepackage{movie15}
\usepackage{subfigure}
\usepackage{wasysym}
\usepackage{moresize}
\usepackage{dsfont}
\usepackage[misc]{ifsym} % for the letter picture before email on last slide

\newcommand{\fT}{\mathcal{T}}
\newcommand{\ZZ}{\mathbb{Z}}
\newcommand{\NN}{\mathbb{N}}
\newcommand{\PP}{\mathbb{P}}
\DeclareMathOperator{\argmin}{argmin}


% Load Theme
\usetheme[noptsans,navigation=true, FB=Mathematik, frametotal=true]{TUKL}
%
\title{Proseminar: Mathematik in Computerspielen}
\subtitle{Delaunay-Triangulierung}
\date[]{16.1.2017} %\\[1ex] OCIP 2016
%\author[Claudia Totzeck]{Claudia Totzeck}
%\institute[]{AG Technomathematik\\FB Mathematik\\TU Kaiserslautern}
%Setze ein Logo auf der Titelseite unten rechts
\renewcommand{\theSecondLogo}{}
\providecommand{\norm}[1]{\lVert#1\rVert}
\newcommand{\dd}{\,\mathrm{d}}
\newtheorem{proposition}{Proposition}

\setbeamercolor*{block title example}{bg=lightgray}
\setbeamercolor*{block body example}{fg= black, bg= white}

\setbeamercolor{block title}{bg=tuklblue,fg=white}

\begin{document}
	\maketitle
	
	
	\begin{frame}{Einleitung}
		
	Primzahltests untersuchen: welche Eigenschaften werden genutzt? \\
	\medskip 
	
	Übertragbarkeit auf Polynome über $\ZZ_q$ bei festem $q \in \PP$?
		
		
		
	\end{frame}
	
	
	\begin{frame}{Fermat}
		
		\begin{block}{Satz von Fermat}
			Ist $p$ eine Primzahl, so gilt $  \forall a \in \NN :  
		\newline	a^{p-1} = 1 \text{ mod } p $
			
			
		\end{block}
		\medskip
		Algebra: $p-1 = |(\ZZ_p)^*|$ \\
		\medskip
		Polynome: $ |(\ZZ_q[x]/f)^*| =q^{deg(f)}-1 $ für irreduzible Polynome
		
		
	
	\end{frame}
	
	\begin{frame}{Fermat}
		\begin{block}{Fermat für Polynome}
			Ist $f$ irreduzibel über $\ZZ_q$, so gilt $  \forall a \in \ZZ_q[x] :
			 \newline a^{q^{deg(f)}-1} = 1 \text{ mod } f$
		\end{block}
	\end{frame}
	
	\begin{frame}{Miller-Rabin}
		
		\begin{itemize}
			\item finde $s,u \in \NN,u $ ungerade mit $p-1=2^su $
			\item wähle $a$ 
			\item teste ob $a^u = 1 \text{ mod } p$
			\item für $1\leq t\leq s, \text{ teste ob } a^{2^s u} = \text{--}1 \text{ mod } p$
			
		\end{itemize}
		
		
	\end{frame}

	\begin{frame}{Miller-Rabin für Polynome}
		\begin{itemize}
			\item finde $s,u \in \NN,u $ ungerade mit $q^{deg(f)}-1=2^su $
			\item wähle $a$ 
			\item teste ob $a^u = 1 \text{ mod } f$
			\item für $1\leq t\leq s, \text{ teste ob } a^{2^s u} = \text{--}1 \text{ mod } f$
			
		\end{itemize}		
	\end{frame}

	\begin{frame}{Schwierigkeiten}
		Laufzeit: 
		
		\begin{itemize}
			\item sehr viele allokationen; gelöst durch in-place rechnen
			\item potenzierung langsam da $u$ oft groß
		\end{itemize}
	\end{frame}





	
	\begin{frame}{Power-Residue Symbol}
		Legendre Symbol für Polynome 
		\begin{block}{Definition}
			Für $d|q-1 \text{ fest},  a,f \in \ZZ_q[x], f \text{ irreduzibel}, f \nmid a :
		\newline 	(a/f)_d = a^{(|f|-1)/d} \text{ mod } f $
			 
		\end{block}
	
		\begin{block}{Reziprozitätsgesetz}
			Seien $f,g$ irreduzible Polynome. Dann gilt:
				$(g/f)_d=(-1)^{deg(f)deg(g)(q-1)/d} (f/g)_d $ 				
		\end{block}
	\end{frame}

	
	\begin{frame}{Jacobi Symbol}
		Verallgemeinerung des Power-Residue Symbols: f muss nicht irreduzibel sein
		
		
		\begin{block}{Reziprozitätsgesetz}
			Seien $f,g$ irreduzible Polynome. Dann gilt:
			noch einfügen!				
		\end{block}
	\end{frame}

	
	\begin{frame}{Power-Residue Test}
		\begin{itemize}
		\item Nutze Reziprozitätsgesetz, um $(a/f)_d $ zu berechnen
		\item vergleiche Ergebnis mit der Definition
		\end{itemize}
	\end{frame}

	
	\begin{frame}{Laufzeit}
		Ein Durchlauf sehr schnell; vergleichbar mit isirreducible
		Problem: gibt oft fälschlicherweise true aus
		abhängig von a
		
	\end{frame}

	
	\begin{frame}{Pocklington}
		Inhalt...
	\end{frame}

	\begin{frame}{Lucas-Folgen}
		Rekursiv definierte Folgen 
	\end{frame}
	
\end{document}